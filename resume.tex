% USE THIS ONE FOR BIGGER SIZE
%\documentclass[12pt]{resume}
% USE THIS ONE FOR A PRETTIER AND STANDARD LAYOUT
\documentclass[margin,line]{resume} 
\usepackage[dvips]{color}

\begin{document}
\name{\Large Sanchit Agarwal}
\begin{resume}

\section{\mysidestyle Contact \\Information}
		5819 Elwood Street Apt\#6                       \hfill \textit{+1-412-499-2271} \\
		Pittsburgh, PA 15232, USA                   \hfill \textit{sagarwa1@cs.cmu.edu}

\section{\mysidestyle Education}
		\textbf{Carnegie Mellon University, School of Computer Science, Pittsburgh} \hfill December 2016 \\
		\textit{Master of Science in Intelligent Information Systems} \hfill  \textbf{4.00/4.00} \\
		\textbf{Indian School of Mines (ISM), Dhanbad} \hfill May 2013 \\
		\textit{Integrated Master of Technology in Mathematics and Computing} \hfill \textbf{9.37/10.00}

\section{\mysidestyle Skills}
   		 	\textbf{Programming: }C,C++,Java,Python,MATLAB,MySQL,Android development

			\textbf{Technologies: }Eclipse, Microsoft Visual Studio, SVN, Perforce, Apache UIMA, Lucene

\section{\mysidestyle Projects \& \\Internships}

		\textbf{Text Based Information Retrieval} \hfill Sept 2015 -- Dec 2015 \\
  		\textit{Carnegie Mellon University, Pittsburgh, USA} \\
		Built Ranked Boolean, Okapi BM25 and Indri retrieval models in Java using Lucene search engine library. Implemented several query operators and added query expansion and learning-to-rank capabilities. 
		Conducted experiments to analyze the performance of the search engine on ClueWeb09 dataset using metrics like Mean Average Precision (MAP) and Precision@N (P@N).

		\textbf{Analysis of fMRI data using Machine Learning} \hfill Nov 2015 -- Dec 2015 \\
  		\textit{Carnegie Mellon University, Pittsburgh, USA} \\
		Performed multi-class kernelized Support Vector Machine (SVM) classsification to categorize the activity being performed by test subjects using their fMRI scans. In another task, used Principal Component 				Analysis (PCA) with ridge regression to predict missing voxels in fMRI scans of test subjects. Used holdout testing and cross-validation to test the model accuracy and prevent any over-fitting. Implementation 			was done in python using scikit-learn library.

		\textbf{Predicting star ratings from customer reviews} \hfill Nov 2015 \\
  		\textit{Carnegie Mellon University, Pittsburgh, USA} \\
		Built using python, a machine learning system from scratch to preprocess data, extract features (bag of words), and learn a multi-class classifier via logistic regression. Implemented stochastic gradient 					descent (SGH) and Batched SGH to train the logistic regression model. The work was done on a dataset from Yelp containing approximately 1 million reviews.

		\textbf{Movie Recommender System} \hfill Oct 2015 \\
  		\textit{Carnegie Mellon University, Pittsburgh, USA} \\
		Developed, in MATLAB, a collaborative filtering based recommender system for movies using a subset of Netflix Prize dataset. Implemented both memory-based and model-based approaches for rating 				prediction. Tried clustering techniques to address cold-start problem. 
	
		\textbf{Clustering on News Articles} \hfill Sept 2015 \\
  		\textit{Carnegie Mellon University, Pittsburgh, USA} \\
		Build a Bipartite Clustering system, a type of reinforcement clustering that produces both document clusters and word clusters simultaneously, in MATLAB. Performed experiments on a subset of TDT4 dataset 		containing around 2000 documents. Evaluated the system's performance using Sum Of Cosine Similarity metric. 

		\textbf{Fuzzy Transform and its Applications} \hfill May 2012 -- May 2013 \\
  		\textit{Indian School of Mines, Dhanbad, India} \\
		Studied and worked on Fuzzy Transform for discrete and continuous functions in one and two \mbox{variables}. Implemented in MATLAB, an approach to solve ordinary differential equations with 						\mbox{ordinary} and fuzzy initial conditions using fuzzy transform. Investigated the approach by comparing it to the exact and Euler's solutions. Also wrote MATLAB code for lossy image compression and reconstruction using fuzzy transform.

        \pagebreak
        
		\textbf{Proofs of Retrievability (POR) in Cloud Storage} \hfill June 2011 -- July 2011 \\
		\textit{University of Calgary, Calgary, Canada} \\
		Implemented a cryptographic scheme (proof of retrievability) that had been proposed for verifiability of data storage in cloud computing environments. The implementation is in C++ and uses several open 				source libraries like GNU Multiple Precision Arithmetic library, OpenSSL and Pairing-Based Cryptography library. 


\section{\mysidestyle Work\\Experience}
    		\textbf{Samsung India Electronics, Noida} \hfill June 2013 -- July 2015\\
		\textit{Lead Engineer}\\
		 Worked on the development of new features and code maintenance of Samsung's voice assistant mobile application, \textit{SVoice}. Managed the java framework and underlying C libraries for speech recognition (to control apps like Music and Camera through voice) and speaker verification (for unlocking a mobile device through voice) features in Samsung smartphones. Also, looked into android code, was involved in \mbox{commercialization} of \textit{SVoice} and OS upgrade projects for wide range of Samsung smartphones and tablets.


\section{\mysidestyle Selected CourseWork}
	
		\textit{Graduate: }Machine Learning, Search Engines, Machine Learning for Text Mining

   		\textit{Undergraduate: }Computer Organization,  Data Structures, Algorithms Design \& Analysis, Operating Systems, Software Engineering, Theory of Computation, DBMS, Probability and Statistics, Statistical Inference, Optimization Techniques, Mathematical Modeling \& Calculus of Variations, Stochastic Process, Abstract Algebra, Linear  Algebra, Numerical Methods, Fuzzy Set Theory and Applications


\section{\mysidestyle Honours and \\Awards}
			 Awarded the \textbf{Best Student Shield} by \textbf{The President of India} at the convocation in May 2014\vspace{2mm}
			 \\Selected twice for the University of Auckland International Summer Scholarship in 2011 and 2012 \vspace{2mm}
			 \\Selected for the University of Queensland Summer Research Scholarship in 2010-11\vspace{2mm}
			 \\Topper of the college for the academic year 2009-2010\vspace{2mm}
			 \\Recipient of the INSPIRE scholarship from the Department of Science and Technology, Govt. of India from 2008-2012\vspace{2mm}
			 \\Placed in the National top 1.5\% in the IIT-JEE 2008 \vspace{2mm}

\section{\mysidestyle Extra Curricular Activities}
		Won \textit{Best Stock Price Data Hack} (Sponsored by FINRA) at PennApps XII, Sept. 2015\vspace{2mm}
		\\Placement Representative, Integrated M.Tech Mathematics \& Computing, Batch 2k13\vspace{2mm}
		\\Hostel Prefect, \textsl{Amber Hostel} (3rd yr. hostel) for the academic session 2010-2011\vspace{2mm}
		\\Selected among top 150 Indian students to attend Level-0 session of Mathematical Training and Talent Search (MTTS), at Sambalpur University, Orissa from June 2009 - July 2009
		\\(MTTS is a summer school organized by the National Board of Higher Mathematics (NBHM))\vspace{2mm}
		\\\textsl{Hobbies}: Reading

\end{resume}
\end{document}
